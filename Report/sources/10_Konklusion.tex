\begin{flushleft}
\doublespacing

I det foreværende projekt, CDIO Final, har vi udarbejdet et virtuelt Matadorspil udviklet i Java. Vi har gennemført projektet ud fra en ambitiøs kravspecifikation og har haft fokus på at levere et program der virker. Det lykkedes, da alle implementerede dele af spillet virker, og som beskrevet i afsnittet om kravspecifikatiner har vi kun fravalgt få features i forhold til spillereglerne. \\\

I CDIO del3 udviklede vi et Junior Monopoly. Arkitekturen fra dette projekt har vi videreudviklet på. Eksempelvis er der ændret på strukturen omkring GUI håndtering. Vi har udviklet en dansksproget udgave, men har inkluderet muligheden for nemt, at erstatte tekster med eventuelle oversatte varianter. Således vil spillet hurtigt kunne leveres til andre lande i andre sprog.\\\

Vi har anvendt mange af de programkonstruktioner, som der er blevet gennemgået igennem semestret og har haft fokus på, at klasserne nedarver mest muligt fra hinanden, at gøre koden overskuelig at læse og lave metoder der kan genbruges. Dog ville vi, som beskrevet i design afsnittet, i en fremtidig version sørge for at minimere dependencies og dermed opnå lavere kobling.\\\

Når vi ser tilbage, oplever vi, at vores planlægning i store træk blev fulgt ret godt i løbet af processen.
Alt i alt har vi som gruppe leveret en ambitiøs og velfungerende virtuel udgave af Matadorspillet, der opfylder alle de krav, der blev sat sammen med kunden.




\end{flushleft}