\begin{flushleft}
\subsection{Funktionelle krav}
\end{flushleft}
\begin{figure}[H]
    \centering
\begin{tabular}{ | c | c | } 
\hline
RS01 & Spillerne begynder på feltet START. \\ 
\hline
RS02 & Der skal være en spilleplade på 40 felter. \\ 
\hline
RS03 & Matadorspillet kan spilles af 2-6 spillere. \\ 
\hline
RS04 & Der skal være to terninger som begge har seks sider. \\ 
\hline
RS05 & Hvis man slår to ens på terningerne rykker man og får lov til at kaste igen. \\
\hline
RS06 & Hvis man kaster to ens tre gange i træk ryger man i fængsel. \\ 
\hline
RS07 & Hver spiller starter med 30.000. \\ 
\hline
RS08 & Hvis en spiller skal betale mere end spillerens samlede ejendommes\\ & værdi går spilleren fallit og ryger ud af spillet. \\ 
\hline
RS09 & Den spiller, der er sidst tilbage på spillepladen vinder. \\ 
\hline
RS10 & Hvis man lander på "Prøv lykken" feltet skal man trække et chancekort\\
& som skal bruges i samme tur. \\ 
\hline
RS11 & En spiller rykker det antal øjne som terningerne viser på spillepladen.\\ 
\hline
RS12 & Vælger en spiller ikke at købe den grund, der landes på, \\
& sættes grunden på auktion.\\ 
\hline
RS13 & Hvis en spiller er i fængsel, kan spiller ikke kræve husleje\\
& på de grunde, der er ejet af spilleren.\\
\hline
RS14 & For at komme ud af fængslet kan en spiller vælge mellem tre valgmuligheder:\\
& betal kaution, slå to ens eller brug løsladelseskort. \\ 
\hline
RS15 & Kommer man ud af fængslet slår man  i samme tur med terningerne og rykker. \\ 
\hline
RS16 & Kommer man ud af fængslet ved at slå to ens, rykker man og får en ekstra tur. \\ 
\hline
RS17 & Spillerne vælger indbyrdes, hvem der starter. \\ 
\hline
RS18 & Hvis man ryger i fængsel modtager man ikke 4000 for at passere START. \\
\hline
RS19 & Det skal være muligt at ændre sprog i spillet. \\ 
\hline
RS20 & Man kan købe huse/hotel, hvis man ejer alle grunde i tilsvarende farve. \\ 
\hline
RS21 & Huslejen stiger, hvis der er bygget huse/hotel i forvejen. \\ 
\hline
RS22 & Der må ikke være mere end et hus forskel mellem grunde af samme farve. \\ 
\hline
RS23 & For at opføre et hotel skal der være fire huse på grunden i forvejen. \\ 
\hline
RS24 & Der kan maksimum være et hotel per grund.\\ 
\hline
RS25 & Når et hotel opføres gives husene tilbage til banken. \\ 
\hline
RS26 & Er man fængslet i tre omgange skal man i fjerde omgang betale 1000, \\
& hvorefter man bliver løsladt.\\
\hline
RS27 & Spillere kan pantsætte ubebyggede grunde for den priser, der står på skødet.\\
\hline
RS28 & Der kræves ikke husleje på en pantsat grund.\\
\hline
RS29 & Lejen fordobles på en grund, hvis spilleren ejer alle grunde i samme farve.\\
\hline
\end{tabular}
\caption{Tabel over prioriterede krav}
\end{figure}
