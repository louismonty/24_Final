\begin{flushleft}
\doublespacing
Vi har udbygget en prioriteret kravliste efter regelsættet der blev udleveret (se Figure \ref{MatadorRules1}-\ref{MatadorRules4} i bilag). Til det første møde med kunden (M1), blev der lavet en skitsering af kravlisten. Under dette møde med kunden blev kravlisten gennemgået, og sat i en prioriteret rækkefølge (se Figure \ref{Requirement list}). \\
\addlinespace[0.5cm]
Vi har nogle få punkter, der er valgt ikke at være med i spillet, grundet den begrænsede mængde tid til rådighed, og at de ikke var vigtige for funktionen eller glæden ved at spille spillet. Disse punkter inkluderer: (1) To måder at spille spillet på, en hurtig og normal måde, (2) at spillerne kan handle indbyrdes med skøder og løsladelseskort, (3) at en spiller bliver bank og auktionerer ejendomme fra spillere der er gået fallit og (4) at der er begrænsede antal huse og hoteller tilgængelige. Alle de fravalgte features vil senere kunne implementeres, hvis kunden ønsker det.  

\end{flushleft}
\begin{figure}[H]
    \centering
\begin{tabular}{ | c | c | } 
\hline
RS01 & En spiller kan købe grunde. \\ 
\hline
RS02 & Spillerne begynder på feltet START. \\ 
\hline
RS03 &  Der skal være en spilleplade på 40 felter.\\ 
\hline
RS04 &  Matadorspillet kan spilles af 2-6 spillere.\\ 
\hline
RS05 & Der skal være to terninger som begge har seks sider. \\
\hline
RS06 &  Hvis man slår to ens på terningerne rykker man og får lov til at kaste igen.\\ 
\hline
RS07 & For at komme ud af fængslet kan en spiller vælge mellem tre valgmuligheder:\\
& betal kaution, slå to ens eller brug løsladelseskort. \\ 
\hline
RS08 &Hver spiller starter med 30.000. \\ 
\hline
RS09 & Den spiller, der er sidst tilbage på spillepladen vinder. \\ 
\hline
RS10 & Hvis man lander på "Prøv lykken" feltet skal man trække et chancekort.\\
\hline
RS11 & Hvis en spiller skal betale mere end hvad spilleren har på kontoen,\\ & går spilleren fallit og ryger ud af spillet.\\
\hline
RS12 & En spiller rykker det antal øjne som terningerne viser på spillepladen. \\ 
\hline
RS13 & Vælger en spiller ikke at købe den grund, der landes på, \\
& sættes grunden på auktion.\\ 
\hline
RS14 & Hvis en spiller er i fængsel, kan spilleren ikke kræve husleje\\
& på de grunde, der er ejet af spilleren. \\
\hline
RS15 & Når man trækker et chancekort skal indholdet udføres med det samme. \\ 
\hline
RS16 &  Spillerne vælger indbyrdes, hvem der starter. \\%29
\hline
RS17 &  Hvis man ryger i fængsel modtager man ikke 4000 for at passere START.\\ 
\hline
RS18 & Det skal være muligt at ændre sprog i spillet.  \\ 
\hline
RS19 & Hvis man kaster to ens tre gange i træk ryger man i fængsel. \\
\hline
RS20 & Man kan købe huse/hotel, hvis man ejer alle grunde i tilsvarende farve. \\
\hline
RS21 &  Huslejen stiger, hvis der er bygget huse/hotel i forvejen.\\ 
\hline
RS22 &  Der må ikke være mere end et hus forskel mellem grunde af samme farve. \\ 
\hline
RS23 & For at opføre et hotel skal der være fire huse på grunden i forvejen.\\ 
\hline
RS24 & Der kan maksimum være et hotel per grund. \\ 
\hline
RS25 &Er man fængslet i tre omgange skal man i fjerde omgang betale 1000, \\
& hvorefter man bliver løsladt. \\
\hline
RS26 &  Kommer man ud af fængslet slår man  i samme tur med terningerne og rykker. \\ 
\hline
RS27 & Spillere kan pantsætte ubebyggede grunde for den pris, der står på skødet. \\
\hline
RS28 & Kommer man ud af fængslet ved at slå to ens, rykker man og får en ekstra tur.\\
\hline
RS29 & Der kræves ikke husleje på en pantsat grund.\\
\hline
RS30 & Lejen fordobles på en grund, hvis spilleren ejer alle grunde i samme farve.\\
\hline
RS31 & Når et hotel opføres gives husene tilbage til banken. \\%
\hline
\end{tabular}
\caption{Tabel over prioriterede krav}
\label{Requirement list}
\end{figure}
