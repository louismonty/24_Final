\begin{flushleft} % sætter tekststarten i venstre marginside.
\doublespacing

I dette projekt har vi arbejdet på, at udvikle en dansk version af spillet Matador ved brug af Java, hvor der er genbrugt visse kodedele fra CDIO del1 til CDIO del3, hvorend hensigtsmæssigt. Den nuværende version af Matadorspillet er udviklet ud fra et virkelig Matadorspils spilleregler, spillekort og spilleplade. Alle spillekort og spillepladen er implementeret, mens få spilleregler er udeladt.

\addlinespace

I projektet er der arbejdet iterativt efter en Unified Process style workflow. Dette vil sige, at der fra start af har været fokus på at implementere centrale use-cases, hvorefter der inkrementelt har været tilføjelser og revidering af use-cases. Derved er projektet altså udarbejdet i stil med en foreberedelses-, etablerings-, konstruktions- og overdragelsesfase (hhv. inception, elaboration, construction og transition på engelsk). Hertil er der udarbejdet analyser indenfor kravspecificering, use-cases, domænet og systemsekvenser. Designet er udviklet ved brug af sekvensdiagrammer og et klassediagram.
\addlinespace
Koden er skrevet ved hjælp af IntelliJ som udviklingsværktøj og Git som versionsstyringsværktøj. Herved er det sikret, at dokumentationen bliver opdateret tilsvarende det aktuelle Matador programs version. Yderligere har vi gjort brug af en Graphical User Interface (GUI), med tilhørende java bibliotek, som er blevet udleveret af kunden.
\addlinespace
I forbindelse med test af produktet er der i projektet udarbejdet JUnit tests og brugertests. Der er primært fokuseret på test, der måler funktionalitet af Matadorspillet.
\end{flushleft}